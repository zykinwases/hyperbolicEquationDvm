\section{Выводы}
\label{sec:Conclusion} \index{Conclusion}

В результате работы над поставленной задачей были разработаны программные средства для решения трёхмерного гиперболического уравнения в прямоугольном параллелепипеде с использованием технологий MPI и OpenMP.
Разработанная программа показала хорошее ускорение при запуске на суперкомпьютерном кластере IBM Polus.

Результаты экспериментов показали, что с увеличением числа точек сетки позволяет увеличить эффективность работы программы.
В этом случае накладные расходы для пересылки данных между процессами уменьшаются по сравнению с вычислительной сложностью задачи.
Более того, с увеличением числа точек сетки увеличивает точность, с которой производятся расчёты.

Исходя из всего вышесказанного, можно сделать вывод, что, несмотря на возможные накладные расходы, использование суперкомпьютеров и технологий параллельного программирования позволяет серьёзно ускорить решение трёхмерного гиперболического уравненияв прямоугольном параллелепипеде.