\section{Эксперименты}
\label{sec:Results} \index{Results}

Для исследования масштабируемости разработанной программы были проведены следующие эксперименты.

\begin{itemize}
    \item В каждом запуске выполнялось 20 шагов по времени.
    \item Параметр $L = L_x = L_y = L_z$ принимал значения $1$ и $\pi$.
    \item Параметр $N$ принимал значения $128^3$ $256^3$ $512^3$.
    \item Программа запускалась в трёх режимах: последовательный, MPI, MPI+OpenMP.
    \item Для запусков MPI и OpenMP использовались $1$, $4$, $8$, $16$, $32$ MPI процессов.
    \item В запусках OpenMP использовались 4 нити.
    \item Каждый вариант запускался по 5 раз с последующим усреднением результатов.
\end{itemize}

Результаты экспериментов приведены в таблицах и на графиках.

Время работы последовательной версии программы и время работы MPI программы на одном процессе примерно совпадают.
Это значит, что на инициализацию инфраструктуры для корректной работы MPI программы занимает достаточно небольшое время.

Во всех случаях MPI программы получились достаточно высокие значения ускорения, что означает хорошую распараллеливаемость алгоритма.

С помощью OpenMP-нитей внутри MPI-процессов было получено дополнительное ускорение, однако оно, в среднем, не превышает 2 (при 4 нитях).
Это может означать то, что на пересылку данных тратится достаточное количество времени.

На рисунках можно видеть графики аналитической функции, вычисленной функции и погрешности вычисления для двух значений $L$.
