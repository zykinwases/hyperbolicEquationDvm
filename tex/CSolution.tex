\section{Численный метод решения}
\label{sec:CSolution} \index{CSolution}

Для численного решения задачи введём на $\Omega$ сетку $\omega_{h\tau} = \overline{\omega_h} \times \omega_\tau$, где
$$
T = T_0,\ L_x = L_{x_0},\ L_y = L_{y_0},\ L_z = L_{z_0},
$$
$$
\overline{\omega_h} = \{ (x_i = i h_x,\ y_j = j h_y,\ z_k = k h_z ),\ i,j,k = 0,1,\dotsc,N,\ h_x N = L_x,\ h_y N = L_y,\ h_z N = L_z \},
$$
$$
\omega_\tau = \{ t_n = n\tau,\ n = 0,1,\dotsc,K,\ \tau K = T \}.
$$

Через $\omega_h$ обозначим множество внутренних, а через $\gamma_h$ -- множество внешних граничных узлов сетки $\overline{\omega_h}$.

Для аппроксимации исходного уравнения \ref{eq:gen} воспользуемся следующей системой уравнений:
$$
\frac{u_{ijk}^{n+1} - 2u_{ijk}^{n} + u_{ijk}^{n-1}}{\tau^2} = \Delta_h u^n,\ (x_i,y_j,z_k) \in \omega_h,\ n = 1,2,\dotsc, K-1,
$$

Здесь $\Delta_h$ -- семиточечный разностный аналог оператора Лапласа:
$$
\Delta_h u^n = \frac{u_{i-1,j,k}^n - 2u_{ijk}^n + u_{i+1,j,k}^n}{h^2} + \frac{u_{i,j-1,k}^n - 2u_{ijk}^n + u_{i,j+1,k}^n}{h^2} + \frac{u_{,j,k-1}^n - 2u_{ijk}^n + u_{i,j,k+1}^n}{h^2}
$$

Приведённая выше схема является явной -- значения $u_{ijk}^{n+1}$ можно явным образом выразить через значения на предыдущих слоях.

Для начала счёта (вычисления $u_{ijk}^2$) должны быть заданы значения $u_{ijk}^0$ и $u_{ijk}^1$, $(x_i,y_j,z_k) \in \omega_h$.
Для начальных условий имеем:
$$
u_{ijk}^0 = \phi(x_i,y_j,z_k),\ (x_i,y_j,z_k) \in \omega_h.
$$
$$
u_{ijk}^1 = u_{ijk}^0 + \frac{\tau^2}{2} \Delta_h \phi(x_i,y_j,z_k),\ (x_i,y_j,z_k) \in \omega_h.
$$

Для граничных условий по $x$ имеем:
$$
u_{0jk}^{n+1} = u_{Njk}^{n+1},\ i,f,k = 0,1,\dotsc,N.
$$

Для граничных условий по $y$ и $z$ имеем:
$$
u_{i0k}^{n+1} = 0,\ u_{iNk}^{n+1} = 0,\ u_{ij0}^{n+1} = 0,\ u_{ijN}^{n+1} = 0.
$$