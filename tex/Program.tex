\section{Программная реализация}
\label{sec:Program} \index{Program}

Для решения поставленной задачи была написана программа на языке C++ с использованием технологий MPI, OpenMP и DVMH.

Программа представляет из себя четыре файла:
\begin{itemize}
    \item equation.cpp -- основная вызывающая программа.
    \item solver.cpp -- последовательное решение задачи.
    \item mpiSolver.cpp -- решение задачи с использованием технологии MPI.
    \item ompMpiSolver.cpp -- решение задачи с использованием технологий MPI+OpenMP.
\end{itemize}

Алгоритм программы:
\begin{enumerate}
    \item Разбиваем область между процессами (в параллельной версии).
    \item Используя формулы из Секции \ref{sec:CSolution} находим значения $u^0$ и $u^1$.
    \item Передаём необходимые значения последнего слоя соседям.
    \item Считаем значения краевых точек по краевым условиям.
    \item Высчитываем значения нового слоя во всех внутренних точках процесса.
    \item Определяем максимальную погрешность на сетке между посчитанным и аналитическим решением.
    \item Повторяем шаги 3-6 для необходимого количества слоёв по времени.
\end{enumerate}

\textbf{Особенности решения задачи с использованием технолии MPI}

\begin{enumerate}
    \item Разбиение области между процессами происходит с помощью функций библиотеки mpi $MPI\_Dims\_create$ и $MPI\_Cart\_create$.
    \item Соседние процессы (с которыми будут производиться обмены) определяются с помощью фукнции библиотеки mpi $MPI\_Cart\_shift$.
    \item Первые и последние элементы слоя, хранящиеся на каждом процессе, если это необходимо, заполняются полученными от соседей элементами.
            Это сделано для сохранения основных вычисляющих функций.
    \item Пересылки необходимых данных производятся в асинхронном режиме.
    \item Максимальная погрешность вычисляется с помощью функции редукции $MPI\_Reduce$.
\end{enumerate}

\textbf{Особенности решения задачи с использованием технолии OpenMP}

Технология OpenMP используется для распараллеливания вычислений краевых условий и внутренних точек на каждом временном слое.
Это возможно, потому что циклы в этих вычислениях не имеют зависимостей по данным. 

\textbf{Особенности решения задачи с использованием технолии DVMH}

В качестве основы для распараллеливания программы с использованием технолии DVMH была взята последовательная версия программы с некоторыми модификациями, а именно с использованием вместо одномерного массива трёхмерного с целью упрощения блочного распределения средставми DVM.

Циклы, отвечающие за вычисление граничных и внутренних значений цикла были расспараллелены с использованием теневых граней.
Цикл подсчёта ошибки -- с использованием редукции максимума.

Все основные вычисления были обёрнуты в отдельный dvm регион для использования распараллеливания в рамках одного хоста.